\documentclass[a4paper, 12pt]{article}

%\usepackage[T2A]{fontenc}
\usepackage[T1]{fontenc}
\usepackage[utf8]{inputenc}
\usepackage[english]{babel}
\usepackage{amsfonts}
\usepackage{amssymb}
\usepackage{amsmath}
\usepackage{amsthm}
\usepackage{tikz-cd}

\newtheorem{thm}{Theorem}
\DeclareMathOperator{\Image}{Im}

\begin{document}
\begin{thm}\label{RelImages}
    Let $X, Y, Z$ be sets, $f:X\to Y,\,g:X\to Z$ are mappings. Then there exists $h: Y\to Z$ 
    such that the diagram
    \[
    \begin{tikzcd}[row sep = tiny]
                                          & Y \arrow[dd, dashrightarrow, "h"] \\
         X\arrow[ru, "f"] \arrow[rd, "g"] & \\
                                          & Z 
    \end{tikzcd}
    \]   
    is commutative iff for all $x_1, x_2\in X$
    \[
        f(x_1)=f(x_2) \Rightarrow g(x_1)=g(x_2) 
    \]
    Moreover, such $h$ is unique iff\, $Y=\Image f$ 
\end{thm}
\begin{proof}
    Lets define $h$ as follows:
    \[
        h(y) = 
        \begin{cases}
            g(f^{-1}(y)), & y\in \Image f\\
            \text{any element}\, z\in Z, & y\notin \Image f 
        \end{cases}
    \]
    The definition is correct because for all $x\in f^{-1}(y)$ $g(x)$ is the same.
\end{proof}

\begin{thm}
    Let $X, Y, Z$ be sets, $f:X\to Z,\,g:Y\to Z$ are mappings. 
    Then there exists $h: X\to Y$ such that the diagram
    \[
    \begin{tikzcd}[row sep = tiny]
        X\arrow[dd, dashrightarrow, "h"] \arrow[rd, "f"] &  \\
                                                         &  Z\\
        Y\arrow[ru, "g"] & 
    \end{tikzcd}
    \]   
    is commutative iff\, $\Image f \subset  \Image g$
\end{thm}
\begin{proof}
    We can choose $h(x) \in g^{-1}(f(x))$ because 
    the preimage is non-empty by condition.
\end{proof}

\begin{thm}
    Let $X, Y$ be sets, $f, g:X\to Y$ are mappings. Then there exists
    a set $O$ and a mapping $\pi: Y\to O$ such that $\pi\circ f=\pi\circ g$
    and for any set $O'$ and mapping $\pi':Y\to O',\; \pi'\circ f = \pi'\circ g$
    there exists unique $h:O\to O'$ such that 
    the diagram is commutative: 
    \[
    \begin{tikzcd}[row sep=tiny]
        &                     &   O\arrow[dd, dashrightarrow, "h"]\\ 
        X\arrow[r, bend left=20, "f"]\arrow[r, bend right=20, "g"{below}]& Y\arrow[ur, "\pi"]\arrow[dr, "\pi'"] & \\
        &                     &   O'\\
    \end{tikzcd}
    \]
\end{thm}
\begin{proof}
    Let $I=\Image f\cup \Image g$, then define
    \[
        O=(I/\thicksim )\sqcup (Y\setminus I)
    \]
    where $\thicksim$ is a minimal equivalence that contains the following 
    relation on $I$:
    \[
        \{(a, b): \exists x\in X\,f(x)=a \land  g(x)=b\}
    \]
    Then lets define
    \[
        \pi(y)=
        \begin{cases}
            [y],&y\in I\\
            y,&y\notin I
        \end{cases}
    \]
    Now $h$ exists and unique according to the theorem \ref{RelImages}
    because it is clear that $\pi(y_1)=\pi(y_2) \Rightarrow
    \pi'(y_1)=\pi'(y_2)$
\end{proof}
\end{document}