\documentclass[a4paper, 12pt]{article}

%\usepackage[T2A]{fontenc}
\usepackage[T1]{fontenc}
\usepackage[utf8]{inputenc}
\usepackage[english]{babel}
\usepackage{amsfonts}
\usepackage{amssymb}
\usepackage{amsmath}
\usepackage{amsthm}
\usepackage{tikz-cd}

\newtheorem{thm}{Theorem}
\DeclareMathOperator{\Image}{Im}

\begin{document}
\begin{thm}\label{RelImages}
    Let $X, Y, Z$ be sets, $f:X\to Y,\,g:X\to Z$ be mappings. Then there exists $h: Y\to Z$ 
    such that the diagram
    \[
    \begin{tikzcd}[row sep = tiny]
                                          & Y \arrow[dd, dashrightarrow, "h"] \\
         X\arrow[ru, "f"] \arrow[rd, "g"] & \\
                                          & Z 
    \end{tikzcd}
    \]   
    is commutative iff for all $x_1, x_2\in X$
    \[
        f(x_1)=f(x_2) \Rightarrow g(x_1)=g(x_2) 
    \]
    Moreover, such $h$ is unique iff\, $Y=\Image f$ 
\end{thm}
\begin{proof}
    Lets define $h$ as follows:
    \[
        h(y) = 
        \begin{cases}
            g(f^{-1}(y)), & y\in \Image f\\
            \text{any element}\, z\in Z, & y\notin \Image f 
        \end{cases}
    \]
    The definition is correct because for all $x\in f^{-1}(y)$ $g(x)$ is the same.
\end{proof}

\begin{thm}
    Let $X, Y, Z$ be sets, $f:X\to Z,\,g:Y\to Z$ are mappings. 
    Then there exists $h: X\to Y$ such that the diagram
    \[
    \begin{tikzcd}[row sep = tiny]
        X\arrow[dd, dashrightarrow, "h"] \arrow[rd, "f"] &  \\
                                                         &  Z\\
        Y\arrow[ru, "g"] & 
    \end{tikzcd}
    \]   
    is commutative iff\, $\Image f \subset  \Image g$
\end{thm}
\begin{proof}
    We can choose $h(x) \in g^{-1}(f(x))$ because 
    the preimage is non-empty by condition.
\end{proof}

\begin{thm}[coequalizer existence]
    Let $X, Y$ be sets, $f, g:X\to Y$ are mappings. Then there exists
    a set $O$ and a mapping $\pi: Y\to O$ such that $\pi\circ f=\pi\circ g$
    and for any set $O'$ and mapping $\pi':Y\to O',\; \pi'\circ f = \pi'\circ g$
    there exists unique $h:O\to O'$ such that 
    the diagram is commutative: 
    \[
    \begin{tikzcd}[row sep=tiny]
        &                     &   O\arrow[dd, dashrightarrow, "h"]\\ 
        X\arrow[r, bend left=20, "f"]\arrow[r, bend right=20, "g"{below}]& Y\arrow[ur, "\pi"]\arrow[dr, "\pi'"] & \\
        &                     &   O'
    \end{tikzcd}
    \]
\end{thm}
\begin{proof}
    Let $O=Y/\thicksim$, where $\thicksim$ is a minimal equivalence that contains the following 
    relation on $Y$:

    \[
        \{(a, b)\in Y^2: \exists x\in X\,f(x)=a \land  g(x)=b\}
    \]
    Then $\pi(x)=[x]_\thicksim$ is a natural projection.
    Now $h$ exists and unique according to the theorem \ref{RelImages}
    because it is clear that $\pi(y_1)=\pi(y_2) \Rightarrow
    \pi'(y_1)=\pi'(y_2)$ and $\pi$ is surjective.
\end{proof}
\begin{thm}[pushout existence]
Let $X, Y, Z$ be sets, $f:X\to Y$ and\\ $g:X\to Z$ be mappings. Then there exists a set $P$
and mappings $\pi_1: Y\to P$ and $\pi_2: Z\to P$ such that the following diagram is commutative:
\[
    \begin{tikzcd}
       X\arrow[r, "f"]\arrow[d, "g"] & Y\arrow[d, "\pi_1"{left}]\\
       Z\arrow[r, "\pi_2"] & P
    \end{tikzcd}
\]
Moreover, for any set $P'$ and 
mappings $\pi'_1:Y\to P'$ and $\pi'_2:Z\to P'$ with the same property there exists a unique $h:P\to P'$ 
such that the diagram
\[
    \begin{tikzcd}
       X\arrow[r, "f"]\arrow[d, "g"] & Y\arrow[d, "\pi_1"{left}]\arrow[ddr, "\pi'_1"] & \\
       Z\arrow[r, "\pi_2"]\arrow[drr, "\pi'_2"{below}] & P\arrow[dr, dashrightarrow, "h"{above}] & \\
       & & P'
    \end{tikzcd}
\]
is commutative.
\end{thm}
\begin{proof}
   Consider the diagram
   \[
        \begin{tikzcd}
            X\arrow[rr, "f"]\arrow[dd, "g"] & & Y\arrow[ld, "i_1"]\arrow[dd, "\pi\circ i_1"] \\
            & Y\sqcup Z\arrow[rd, "\pi"] & \\
            Z\arrow[ru, "i_2"]\arrow[rr, "\pi\circ i_2"] & & O
        \end{tikzcd}
   \]
   where $\pi$ is a coequalizer of $i_1\circ f$ and $i_2\circ g$. It is clear
   that $O$ is a pushout.
\end{proof}


\end{document}