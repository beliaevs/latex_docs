\documentclass{article}

\usepackage{amsfonts}
\usepackage{amssymb}
\usepackage{amsmath}
\usepackage{amsthm}
\usepackage{tikz-cd}

\newtheorem{thm}{Theorem}
\DeclareMathOperator{\Image}{Im}

\begin{document}

\begin{thm}
    Let $X, Y, Z$ be sets, $f:X\to Y,\,g:X\to Z$ are mappings. Then there exists $h: Y\to Z$ 
    such that the diagram
    \[
    \begin{tikzcd}[row sep = tiny]
                                          & Y \arrow[dd, dashrightarrow, "h"] \\
         X\arrow[ru, "f"] \arrow[rd, "g"] & \\
                                          & Z 
    \end{tikzcd}
    \]   
    is commutative iff for all $x_1, x_2\in X$
    \[
        f(x_1)=f(x_2) \Rightarrow g(x_1)=g(x_2) 
    \] 
\end{thm}
\begin{proof}
    Lets define $h$ as follows:
    \[
        h(y) = 
        \begin{cases}
            g(f^{-1}(y)), & y\in \Image f\\
            \text{any}, & \text{otherwise} 
        \end{cases}
    \]
    The definition is correct because for all $x\in f^{-1}(y)$ $g(x)$ is the same.
\end{proof}

\begin{thm}
    Let $X, Y, Z$ be sets, $f:X\to Z,\,g:Y\to Z$ are mappings. 
    Then there exists $h: X\to Y$ such that the diagram
    \[
    \begin{tikzcd}[row sep = tiny]
        X\arrow[dd, dashrightarrow, "h"] \arrow[rd, "f"] &  \\
                                                         &  Z\\
        Y\arrow[ru, "g"] & 
    \end{tikzcd}
    \]   
    is commutative iff\, $\Image f \subset  \Image g$
\end{thm}
\begin{proof}
    We can choose $h(x) \in g^{-1}(f(x))$ because 
    the preimage is non-empty by condition.
\end{proof}

\end{document}