\documentclass[14pt, a4paper]{extarticle}
\usepackage{amsfonts}
\usepackage{amsmath}
\usepackage{amssymb}
\usepackage{amsthm}
\usepackage[T2A]{fontenc}
\usepackage[english,russian]{babel}

\pagestyle{empty}
\linespread{1.5}

\theoremstyle{definition}
\newtheorem{task}{Задача}

\begin{document}

\begin{task}
    Пирог поделен на 12 равных кусков. Скольким гостям можно раздать все куски пирога так, чтобы всем досталось поровну?
    Напиши все возможные ответы. 
\end{task}

\begin{task}
    В коробке лежат 7 конфет. Скольким детям можно их раздать, чтобы никого не обидеть?
    Напиши все возможные ответы.
\end{task}

\begin{task}
    Ваня и Дима вместе съели 40 пельменей, причем Ваня съел на 6 больше, чем Дима.
    Сколько пельменей съел каждый ребенок?
\end{task}

\begin{task}
    Резиновый квадрат со стороной 4 сантиметра растянули в 3 раза, и теперь длина
    его стороны стала равна 12 сантиметрам. Во сколько раз увеличился его
    периметр? А площадь?
\end{task}

\begin{task}
    Алиса выпила волшебный пузырек и ее рост увеличился в 4 раза. Во сколько
    раз увеличился ее вес?
\end{task}

\begin{task}
    Британские единицы длины довольно необычны. В одном ярде --- 3 фута. 
    В одном футе --- 12 дюймов. Сколько дюймов в ярде? Сколько сантиметров в ярде,
    если в одном дюйме 25 миллиметров?
\end{task}

\end{document}