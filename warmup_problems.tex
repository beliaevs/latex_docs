\documentclass[14pt, a4paper]{extarticle}
\usepackage{amsfonts}
\usepackage{amsmath}
\usepackage{amssymb}
\usepackage{amsthm}
\usepackage[T2A]{fontenc}
\usepackage[english,russian]{babel}

\pagestyle{empty}
\linespread{1.5}

\theoremstyle{definition}
\newtheorem{task}{Задача}

\begin{document}

\begin{task}
   Гномик пошел ловить ершей. Он поймал 5 ершей плюс половину всего улова. Сколько ершей он поймал?
\end{task}

\begin{task}
    Бегемот тяжелее жирафа, а жираф тяжелее носорога. Кто тяжелее: бегемот или носорог?
\end{task}

\begin{task}
На острове рыцарей и лжецов живут два брата-близнеца: Сеня и Веня. Каждый из них
либо рыцарь, либо лжец. В день, когда им исполнилось 10 лет, Сеня сказал, что 
ему не меньше девяти лет, а Веня, что ему меньше десяти. Кто Сеня и кто Веня --- рыцари
или лжецы?
\end{task}

\begin{task}
У капитана Кука есть попугай Полли, который накануне бури всегда чихает. Полли только что чихнул.
Капитан Кук говорит: <<Мой попугай чихнул, значит завтра будет буря>>.
Верно ли умозаключение капитана? Обязательно ли завтра будет буря?
\end{task}

\begin{task}
   В доме десять этажей. На первом этаже живет один человек, на втором --- два, на третьем --- три
   и так далее. На каком этаже лифт останавливается чаще всего?
\end{task}

\begin{task}
    Лифт поднимается с первого этажа до третьего за 3 секунды. За какое время он поднимется с первого этажа до девятого?
\end{task}
\end{document}