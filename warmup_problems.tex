\documentclass[14pt, a4paper]{extarticle}
\usepackage{amsfonts}
\usepackage{amsmath}
\usepackage{amssymb}
\usepackage{amsthm}
\usepackage[T2A]{fontenc}
\usepackage[english,russian]{babel}

\pagestyle{empty}
\linespread{1.5}

\theoremstyle{definition}
\newtheorem{task}{Задача}

\begin{document}

\begin{task}
    В семье семь братьев. У каждого брата одна сестра. Сколько всего детей в этой семье?
\end{task}

\begin{task}
    Некто положил две монеты в два кошелька так, что в одном кошельке оказалось в два раза больше монет,
    чем в другом. Как он это сделал?
\end{task}

\begin{task}
Когда Женя получает пятёрку по русскому языку, она всегда приходит домой в отличном настроении.
Сегодня Женя пришла домой в отличном настроении. Значит ли это, что 
она получила пятёрку по русскому языку?
\end{task}

\begin{task}
У капитана Кука есть попугай Полли, который накануне бури всегда чихает. Полли только что чихнул.
Капитан Кук говорит: <<Мой попугай чихнул, значит завтра будет буря>>.
Верно ли умозаключение капитана? Обязательно ли завтра будет буря?
\end{task}

\begin{task}
    В абсолютно темной комнате стоит стеклянная ваза, в которой 10 чёрных и 12 белых шаров.
    Какое наименьшее число шаров надо вынуть из сосуда, чтобы можно было с уверенностью 
    сказать следущее:
    \begin{itemize}
        \item среди вынутых шаров есть пара шаров одного цвета;
        \item среди вынутых шаров есть пара чёрных шаров?
    \end{itemize}
\end{task}

\begin{task}
    Бобёр распили два бревна на поленья. Всего он сделал 40 распилов.
    Сколько получилось поленьев?
\end{task}
\end{document}