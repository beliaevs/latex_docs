\documentclass[a4paper,12pt]{article}
\usepackage{amsmath}
\usepackage{amssymb}
\usepackage[T2A]{fontenc}
%% \usepackage[utf8]{inputenc}
\usepackage[english,russian]{babel}
\linespread{1.3}

\begin{document}
    \section{О чем эти заметки?}
    Это --- рассказ о том, какие бывают числа. Натуральные, целые, отри\-ца\-тель\-ные, дробные... Числа каждого вида нужны для какой-то цели, решают задачу,
    которая без их помощи решается плохо.

    \section{Натуральные числа}
    $1, 2, 3, \ldots$, такие числа знакомы всем. Они нужны для подсчета количеств.
    Над числами можно выполнять
    разные действия. Сложение и умножение --- самые хорошие действия,
    они выполнимы всегда. 
    
    \subsection{Вычитание}
    Вычитание натуральных выполнимо не всегда: если вычитаемое больше 
    уменьшаемого, то вычесть не удастся. Например: у Васи было 4 яблока. 
    Затем 6 он отдал Пете... Стоп-стоп, это ведь невозможно!
    Если же речь идет не о подсчете количеств, то такое вычитание 
    все-таки может иметь смысл: температура воздуха на улице
    4 градуса, затем похолодало на 6 градусов. Какой стала
    температура? Это значение температуры нельзя выразить \emph{натуральным}
    числом.

    \subsection{Деление}
    Кучу из 10 яблок нельзя разделить между тремя детьми поровну.
    Зато \emph{деление с остатком} выполнимо всегда. Например,
    из 10 яблок можно 9 раздать детям, каждому по 3, а 
    одно --- оставить. Подумай, почему любое количество яблок 
    можно разделить на 3 равные кучи так, что оставить придется не 
    больше двух яблок.
    \[
    a = b\cdot n + r, \; 0 \leqslant r < b 
    \]   
    Эта запись означает, что при делении $a$ на $b$ получается $n$, а 
    $r$ - \emph{в остатке}.
    Как было бы хорошо уметь делить любое число на любое!
    
    \subsection{Четные и нечетные числа}
    \emph{Четное} число делится на 2, \emph{нечетное} --- не делится. Нечетное число 
    при делении на 2 дает в остатке 1. Интересно, что четность суммы, разности,
    произведения чисел можно легко узнать:

    \vspace*{\baselineskip}
    
    \begin{tabular}{r|cc}
        + & 0 & 1 \\
        \hline
        0 & 0 & 1 \\
        1 & 1 & 0
    \end{tabular}
    \qquad
    \begin{tabular}{r|cc}
        $\times$ & 0 & 1 \\
        \hline
        0     & 0 & 0 \\
        1     & 0 & 1 \\
    \end{tabular}

    \vspace*{\baselineskip}

    То есть сумма чисел одной четности --- четна, а разной четности --- нечетна.
    Произведение нечетно только если оба множителя нечетны. Подумай, почему
    так получается.

    \subsection{Десятичная запись чисел}
    Для записи чисел используют \emph{цифры}, также как для написания слов --- буквы.
    Положение цифры в записи числа очень важно, сравни, например, 105 и 150, 17 и 71.
    Запись 1234 обозначает число, в котором 1 тысяча, 2 сотни, 3 десятка и 4 единицы.
    Число 10 играет особую роль, ведь сотня --- это 10 десятков, а тысяча --- 10 
    сотен. 
    \begin{align*}
        abcd &= 1000\cdot a + 100\cdot b + 10\cdot c +  d = \\
             &= 10^3\cdot a + 10^2\cdot b + 10\cdot c + d  
    \end{align*}

    \subsection{Системы счисления}
    Число 10 используется в нашей записи чисел потому, что считать десятками людям
    удобно --- при счете можно загибать пальцы, а пальцев на руках у людей 
    как раз десять. Пальцы на ногах наши предки тоже использовали для счета, поэтому в названиях
    чисел есть следы \emph{двадцатеричной системы счисления}: числа от 11 до 20 называются
    одним словом, а 21, 22, 23 --- уже двумя.

    \subsection{Двоичная система счисления}
    В десятичной системе нужно использовать десять различных цифр: 0, 1, 2, ..., 9.
    В \emph{двоичной системе} можно обойтись всего двумя: 0 и 1. Из-за своей простоты
    этот способ представления чисел используется в компьютерах: для хранения в 
    компьютерах используются ячейки памяти, которые могут быть или пусты, или 
    заполнены, что подходит для хранения двух цифр. Вот примеры записи чисел в 
    двоичной системе:
    \begin{align*}
        1 &= 1_2  &  2 &= 10_2  &  3 &= 11_2  &  4 &= 100_2 \\
        5 &= 101_2  &  6 &=  110_2 & 7 &= 111_2  &  8 &= 1000_2\\
        9 &= 1001_2  &  10 &= 1010_2  &  11 &= 1011_2  &  12 &= 1100_2\\    
    \end{align*}
    Например, число записанное в двоичной системе как $abcde_2$, где
    будет равно
    \begin{align*}
        abcde_2 &= a\cdot 2^4 + b\cdot 2^3 + c\cdot 2^2 + d\cdot 2 + e\\
                &= 16\cdot a + 8\cdot b + 4\cdot c + 2\cdot d + e
    \end{align*}
    где цифры a, b, c, d, e могут быть равны только нулю или единице.
    Получается, что любое число можно получить как сумму различных
    степеней двойки: $1, 2, 4, 8, 16, \ldots$ 
    
    
\end{document}