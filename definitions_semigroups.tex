\documentclass{article}

\usepackage{amsfonts}
\usepackage{amssymb}
\usepackage{amsmath}
\usepackage{amsthm}

\newtheorem{definition}{Definition}
\newtheorem{thm}{Theorem}

\begin{document}
\begin{definition}
    Let $(S, \cdot)$ be a set with binary operation. 
    Then $S$ called a semigroup, if operation $\cdot$ is accociative, namely,
    \[
        (a\cdot b)\cdot c= a\cdot (b\cdot c)
        \] 
    for all elements $a, b, c$ from $S$.
\end{definition}

\begin{thm}
    Let $S$ be a semigroup with the property:
    \[
        aS = S,\;Sa=S    
    \]
    for all $a\in S$. Then S is a group.
\end{thm}
\begin{proof}
    We show that $S$ has a left unit and left inverse.

    Choose an arbitrary element $a\in S$. There exists $e_a\in S$ such that $e_a\cdot a = a$.
    Lets check that $e_a$ is a left unit. For every $x\in S$ there exists 
    $y\in S$ such that $x = a\cdot y$. Then
    \[
        e_a\cdot x = e_a\cdot(a\cdot y) = (e_a\cdot a)\cdot y=a\cdot y=x
        \]
    Therefore, $e_a$ is a left unit.

    If $x\in S$, then there is an element $y$ such that $y\cdot x=e_a$, i.e. $y=x^{-1}$
    is a left inverse for $x$.
\end{proof}
\end{document}